\documentclass[10pt,a4paper,twocolumn]{article}
\usepackage[utf8]{inputenc}
\usepackage{cite}
\usepackage{titling}
\usepackage[hidelinks]{hyperref}
\usepackage{csquotes}
\usepackage[margin=1in]{geometry}

\author{Reid McIlroy-Young}
\title{Short Paper 1\\Ethics of T3}
\date{October 10, 2016}

\begin{document}
\setlength{\droptitle}{-60pt}
\maketitle

As of the writing of this report the data set gathered from 1640 of the 2006 incoming students at Harvard College\cite{zimmerblog} by K. Lewis, \textit{et. al.}\cite{t3} is not available\cite{t3site}, and it likely will never be available. The moral outrage\cite{zimethics} that surrounds the data gathering\cite{t3} and successive studies\cite{t3followup1}\cite{t3followup2} has insured that the data will never again be released. That said, as a research interested in social networks if I were given an opportunity to gain access to the \textit{Tastes, Ties, and Time} (T3) it is unlikely I would decline the offer. The data are unmatched in length and scope, a pinnacle that will likely never be reached again by public researchers. As I will discuss below a contemporary institutional review board (IRB) would never allow a study like Lewis's to be run. The modern framework to discuss the ethics of T3 we will use is that of the Menlo Report\cite{menlo}, which was written two years after the completion of T3 and is intended to be a version of the Belmont Report\cite{belmont} for digital research\cite{bitbybit}.

\subsubsection*{Respect for Persons}

One of the authors of T3 responded to criticisms\cite{zimmerblog} of their methods by comparing it to surveillance in a public setting.

\begin{displayquote}
\textit{Could you require that someone sitting in a public square, observing individuals and taking notes on their behavior, would have to ask those individuals’ consent in advance? We have not accessed any information not otherwise available on Facebook.}\cite{zimmerblogcomment}
\end{displayquote}

If the data were truly public, this argument would still not be fully in line with the \textit{Respect for Persons} mandated of the Menlo report. They did not obtain informed consent nor did they attempt to remove those with diminished autonomy. 

At the time studies trawling public data on social media was not commonly known to exist so the participants would not have suspected it might be happening, unlike the example given of someone making observations in public, which is a phenomena in the Zeitgeist and thus people can be considered to implicitly giving consent as part of the social contract. By not allowing for even implicit consent the T3 researchers violate the autonomy of the participants. It is also worth noting that they could have obtained consent from every participant without much effort as Harvard College provided them with the email addresses of each of the subjects\cite{t3}. They could have contacted the participants and obtained consent, and they did not. 

One final note: the authors likely did not know that their ripping techniques were allowing them access to data not intended to be public\cite{zimmerblog}, usage of this data is against the \textit{Respect for Persons} principle as it violates there autonomy, privacy and any semblance of consent. But, as effects the authors were aware of also violated the principles I think looking at those is more pedagogical. 

\subsubsection*{Beneficence}

To the best of my knowledge no harm to any individual can be directly traced to the T3 data gathering or any of the subsequent usage, but, that is due to luck. The authors did not make any real attempts and minimizing risk; all they did to the data is remove names\cite{t3}, nothing else. In fact they added more information to the set then was available from the scraping of Facebook, they were able to add data about the participants' race, gender and socio-economic background even when it was not given\cite{t3}. This data along with knowledge that the participants are from Harvard College starting in 2006 is enough to de-anonymize a vast majority of them\cite{bitbybit}. Thus, if the data were released publicly it would be de-anonymized and its scope and breadth mean it would almost certainly cause some people harm and many embarrassment. The researchers have since ceased distribution of the data but the risk still remains. The benefits for the participants are virtually nonexistent so with such a large risk the ethics of T3 are dubious from a \textit{Beneficence} standpoint.

\subsubsection*{Justice}

The participants of T3 were not a vulnerable group as a whole, although some individuals may have been (as discussed above). But their efforts in; creating and maintaining their Facebook pages were in no way reimbursed by the study. The researchers used the efforts of 1640 students for their own benefit, and a marginal societal benefit, and gave nothing back to the participants. The burden of the study thus obviously was not fairly distributed.

\subsubsection*{Respect for Law and Public Interest}

It appears that the T3 study was done legally and with due diligence. The researchers obtained permission from Facebook,  Harvard College and an IRB\cite{t3}. They thus were in full compliance with the law; additionally they were very transparent about what they had done even to the point of offering the data set to others. Their actions after publication show at least some effort at accountability, which is admirable, they addressed the criticism\cite{zimmerblogcomment} and even decided they were wrong, changing their behaviour to minimize harm.

\subsubsection*{Conclusion}

Since T3 norms around privacy on social media have changed and the chilling effects of the publication of research and surveillance of social media\cite{chillingeffects} have been observed. So even if researchers were able to make the same recordings as T3 there would still be significant differences. Although as discussed above the likelihood of this happening is minuscule. Thus the data set is unique and due to it's unavailability\cite{t3site} likely not fully studied. I think that if given the opportunity I would choose to obtain and study the data set but I would not like for it to be made publicly available. My own usage of the T3 data can be defended within the framework of the Menlo Report's principles. 

It has been over a decade since the data collection was started so the impact on the individuals by using the data is much lower than the collection 10 years ago. Also there has been time since the T3 article for the participants to learn of and speak out against the report, it is highly likely that at least some are aware of being in the study now. So if I were to put some effort into removing individuals who could be impacted negatively by the data's use I believe I would satisfy the \textit{Respect for Persons} principle. The \textit{Beneficence} principle would similarity be met by this filtering, but there is also a risk of the data being leaked by my usage that should be addressed. I believe I would be able to maintain control of the data set and not cause harm with it, but there is an increase in risk caused by its dissemination that is worth considering. Since the data have already been collected there is no additional work required of the participants so I think the \textit{Justice} principle is satisfied. Similarly the T3 team has already done what was necessary to make the data collection legal so, as long as my own IRB and their obligations were met, I think my obtaining of the T3 data set would be legal and ethical under the \textit{Respect for Law and Public Interest} principle. 

\bibliography{reid-paper-1}{}
\bibliographystyle{unsrt}

\end{document}